\documentclass[a4paper,10pt]{article}
\usepackage[utf8]{inputenc}
\usepackage{amsmath}
\usepackage{a4wide}


%opening
\title{Cheat Sheet - Uncertainty calculation}
\author{Sebastian Dick}
\begin{document}
\maketitle

\section{Definitions}

Let us say you measure the length of your table to be

\[
l = 50.5 \pm 0.1 \text{ cm.}
\]


\begin{itemize}
 \item The \textbf{best value} $X$ is the value you measured and given in our example as $50.5 \text{ cm}$
 \item The \textbf{absolute error} $\Delta X$ in our example is $ \Delta l = 0.1 \text{ cm} $. Note that the absolute value has the same units as the best value. 
 If the best value is scaled by a factor of $a$ then also the absolute error is multiplied by this factor
 \[
  \Delta(aX)= a \Delta X \\
 \]
 \item The \textbf{relative error} is given by the absolute error divided by the best value and has therefore no units.
 \[ 
  \Delta_{rel}X = \frac{\Delta X}{X}
 \]
 In our example $\Delta_{rel}l =\Delta l / l = 50.5 \text{ cm} / 0.1 \text{ cm } =  0.002$ 

\end{itemize}

\section{Random error}
Imagine you measure the length of the table \textbf{multiple times}. Your data set (measured lengths) is given by

\begin{center}
\begin{tabular}{|c|}
 \hline
 Length / cm \\
 \hline
 $50.1$  \\ 
 $50.2$  \\ 
 $50.0$  \\ 
 $50.5$  \\
 \hline
 
\end{tabular}
\end{center}
Let $x_i$ denote the different measurement outcomes ($i$ is just an index to label the measurements) and N the number of measurements.  
\begin{itemize} 
 \item The \textbf{mean} or \textbf{average} of the data set can be calculated as 
 \[ 
  \overline{x} = \frac{\sum_i x_i}{ N}.
 \]
 or in our case
 \[
   \overline{l} = \frac{50.1+50.2+50.0+50.5}{4} \text{ cm} = 50.2 \text{ cm}
 \]
 \textit{TIP: In Excel you can use the AVERAGE(data) function}
 \item The \textbf{standard error} or the error of the average is then given by
 \[
  \Delta \overline{x} = \sqrt{ \frac{\sum_i (x_i - \overline{x})^2}{N(N+1)}}
 \]
 for our example
 \[
  \Delta \overline{l} = \sqrt{\frac{(50.1-50.2)^2 + (50.2 - 50.2)^2 + ... + (50.5-50.2)^2}{20}} \text{ cm} = 0.11 \text{ cm}
 \]
 The standard error gives you information about how likely it is that your measured \textbf{mean value} is close to the \textbf{true value}. 
 For example, in our experiment the true length of the table is between $50.1$ cm and $50.3$ cm (i.e. the mean value $\pm$ the standard error)
 with a probability/certainty of about $68 \%$

 \textit{TIP: In Excel, use STDEV(data)/SQRT(COUNT(data))}
\end{itemize}

\section{Error propagation}

Imagine you measure the width of the table to be 
\[
 w = 30.2 \pm 0.2 \text{ cm}
\]
and you want to calculate the circumference and area of the table. How does one combine the two uncertainties. Answer: you use the 
\textbf{geometric mean}, which means ``add the squared values and then take the square root''. 
\begin{itemize} 
 \item \textbf{Addition/Subtraction}
 
 Adding/Subtracting the measured values $X$ and $Y$, the combined/propagated error 
 is given by the geometric mean of the \textbf{absolute errors} $\Delta X$ and $\Delta Y$.
 
 \begin{align*}
  Z &= X+Y \\
  \Delta Z &= \sqrt{(\Delta X)^2 + (\Delta Y)^2} \\  
 \end{align*}
 
 So in our example, the circumference of the table and its error is given by
 \begin{align*}
  c &= (30.2 + 50.5) \text{ cm} = 80.7 \text{ cm} \\
  \Delta c &= \sqrt{(0.1)^2+(0.2)^2} \text{ cm}= 0.2 \text{ cm}
 \end{align*}

  \item \textbf{Multiplication/Division}
  
  Multiplying/Dividing $X$ and $Y$, one needs to calculate the geometric mean of the \textbf{relative errors} to obtain the 
  combined relative error
  
  \begin{align*}
  Z &= X \cdot Y \\
  \Delta_{rel} Z &= \frac{\Delta Z}{Z} = \sqrt{\left(\frac{\Delta X}{X}\right)^2 + \left(\frac{\Delta Y}{Y}\right)^2} \\  
  \Delta Z &= (\Delta_{rel}Z) \cdot Z
  \end{align*}
  
  Thus, the area of the table and its relative and absolute error are given by
  
  \begin{align*}
  A &=  30.2 \text{ cm}  \cdot 50.5 \text{ cm} = 1525.1 \text{ cm}^2 \\
  \Delta_{rel} A &= \sqrt{\left(\frac{0.2}{30.2}\right)^2 + \left(\frac{0.1}{50.5}\right)^2} = 0.0069 \\ 
  \Delta A &= 0.0069 \cdot 1525.1 \text{ cm} ^2 = 10 \text{ cm} ^2   \\
  \end{align*}
  Therefore our final result is 
  \[
   A = (15.3 \pm 0.1) \times 10^{2} \text{ cm}^2 = 15.3 \pm 0.1 \text{ dm}^2
  \]
  
  \item \textbf{Powers}
  
  If you are dealing with powers, the relative error of $Z = X^n$ is just the relative error of $X$ multiplied by the absolute value of the
  exponent n ( n does not have to
  be positive or even an integer).
  
  \[
   \frac{\Delta Z}{Z} = |n| \times \frac{\Delta X}{X}
  \]

  

  
  
\end{itemize}


\end{document}
